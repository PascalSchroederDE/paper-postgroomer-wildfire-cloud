%!TEX root = ../dokumentation.tex

\pagestyle{plain}

\iflang{de}{%
\addchap{\langabstractDE} % Text für Überschrift

In der modernen Welt der Informatik und künstlicher Intelligenz gewinnen Daten zunehmend an Bedeutung. Diese Daten werden benötigt, um Machine Learning Systeme zu trainieren sowie für bessere Entscheidungsfindung durch Geschäftsanalysen. In den letzten Jahren war dabei ein starker Anstieg an Echtzeit-Analysen an Stelle der herkömmlichen Analysen, welche wesentlich mehr Zeit beanspruchen, zu vernehmen.

Dafür hat IBM Research einen Prototypen entwickelt, welches die schnellen Transaktion von ``transactional processing'' mit den Analysefunktionen von ``analytical processing'' verbindet. Ein solches System wird ``Hybrid Transactional Analytical Processing'' genannt. Um sowohl Transaktionen als auch Big Data Analysen zu gewährleisten müssen die Daten in verschiedenen Formen vorliegen, was bedeutet, dass diese transformiert werden müssen. In IBMs Prototypen namens ``Wildfire'' gibt es dafür zwei Phasen - die Grooming-Phase sowie die Postgrooming Phase.

In dieser Arbeit wird zunächst Wildfire und ihre zugrunde liegenden Technologien beschrieben und erklärt. Desweiteren werden Leistungs Tests mit dem Wildfire postgroomer durchgeführt. Um die Leistung und Verfügbarkeit zu steigern, wurde Wildfire in ein Kubernetes Cluster aufgesetzt. Die Erklärung von Kubernetes und dessen Funktionen wird ebenso ein Teil dieser Arbeit sein.

Der letzte Teil handelt um eine weitere Bewegung in der Informatik - der Zug zur Cloud. Für erhöhte Effizienz ziehen immer mehr Firmen vom physischen Maschinen zu Cloud Maschinen um - so auch IBM Research. Das Ziel dessen ist es Rechenleistung in der Cloud zu konzentrieren. In folgender Arbeit werden diese Cloud Systeme evaluiert, um diese als eventuellen Ersatz der physischen Geräte für den Wildfire postgroomer in Betracht zu ziehen.

}