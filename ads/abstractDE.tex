%!TEX root = ../dokumentation.tex

\pagestyle{plain}

\iflang{de}{%
\addchap{\langabstractDE} % Text für Überschrift

Ob Alexa, Google Assistant, Siri oder Cortana - die Begeisterung für digitale Sprachassistenten ist groß und nimmt stetig zu. Bereits 2016 nutzten über 390 Millionen Menschen weltweit digitale Assistenzsysteme. Für 2018 wird ein Anstieg auf bis zu eine Milliarde Nutzer erwartet. Die Einsatzmöglichkeiten beschränken sich dabei jedoch nicht nur auf ein Smartphone oder intelligente Lautsprecher. Kognitive Assistenten sind nahezu überall im Alltag einsetzbar - ob zuhause, unterwegs oder auf der Arbeit. Im Verbund mit dem Internet of Things (IoT) und entsprechender Sensorik wird in Zukunft wohl kaum einer an diesen Systemen vorbei kommen. Doch was genau steckt hinter den digitalen Assistenten? In welchen Bereichen kann man sie einsetzen, wie funktioniert die Technik dahinter und wie sind solche Systeme aufgebaut? 

Um diese Themen kümmert sich die folgende Abhandlung. Dazu behandelt es zunächst grundsätzlich kognitive Assistenten, deren Geschichte sowie Einsatzmöglichkeiten und klärt den Zusammenhang zum IoT auf. Im weiteren Verlauf wird näher auf den möglichen Aufbau eines persönlichen, kognitiven Assistenzsystems eingegangen und sich näher auf den Einsatz im Bereich Automotive fokussiert. Dieser wird mit anderen Einsatzmöglichkeiten verglichen.

Abschließend soll geklärt werden, wie ein solches System aufgebaut ist und wie ein Entwicklungsvorgang von statten geht, um so die hinter stehende Technik näher beleuchten und erläutern zu können. Ebenso soll es Gemeinsamkeiten und Unterschiede zwischen Assistenzsysteme im Auto und anderen Bereichen aufzeigen und so Erkenntnisse und Nutzen kooperativer Arbeit für alle Industrien gewinnen.

}