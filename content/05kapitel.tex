%!TEX root = ../dokumentation.tex

\chapter{Discussion}

\section{Evaluation of advantages on cloud-based usage of Wildfire}

At the beginning of this report one of the objectives was to evaluate cloud devices compared to physical devices and figure out their advantages and disadvantages.

The problem of Cloud devices is the constant need of a fast and reliable internet connection. As soon as this fails, no calculations can be made on the Cloud, because there is no Connection to those devices.

This connection over the internet also slights the risk of a cyber attack, which could try to overload the connection or even steal data. This is easier if the devices communicate over the internet instead of just a local network like it is usual with physical devices.

Another important factor for evaluating Cloud devices compared to physical ones is the performance. Chapter 4.2 therefore showed the results of performance comparisons between the Cloud machines and the physical devices called ``Loud''.

Those results showed a significant difference on 8GB and 8 core devices, but only minimalistic differences in all other setups.

The reason for this could be, that for this small amount of data 8GB of memory are enough to run on full power, which results in no significant performance improvements after 8GB of memory can be used. But because for the 8GB Cloud device the 8GB memory couldn't be completely used, because there was some memory reserved for the system, the tests on this configuration has been significant slower.

Those tests showed that the Cloud devices are comparable to physical devices and can be a good replacement for those in the future.

Other advantages for using Cloud Devices instead of physical devices are caused by its centralization of computing power in the Cloud instead of local resources. For example this leads to High Availability, because the Cloud is built with several, redundant machines, which replaces each other if one machine fails. This means, that the virtual devices are always available in the opposite of physical devices, which can't be used any more as soon as they fail, because there is no redundant machine ready for replacing it.

Another advantage are cost savings, because the computing power can be centralized on one computer center and not every location needs its own servers for providing the necessary computing power. Thereby everybody can share the resources and can only claim the power it actually needs, instead of always having more resources than they normally need, because maybe they once need more computing power than usual. This dynamically allocation of computing power increases the cost efficiency of the computing power a lot.

All in all the Cloud devices are a good, possible replacement for a high amount of physical devices - especially in the future. Through increasing speed of the internet connections this problem should become even smaller than today. If also the problem of possible security attacks can be kept low, there should be no reason for not moving to the Cloud devices, as long as they still provide the same performance as the physical devices. Still, there should never be worker on Cloud only, because some physical, local computing powers will always be necessary for example for high confidential calculations or in case of a connection problem, but the main power could possibly move to the Cloud in the future.

\section{Evaluation of running Wildfire post-groomer on cluster and next steps}

In general using a cluster system has many advantages, which has been described in chapter 1.1. How far those advantages hold for Wildfire post-groomer as well could not be exactly determined, because Spark 2.0.2 dependencies prevented a functional testing. This means the main objective to deploy a runnable version of the Wildfire post-groomer could not be achieved. Instead alternatives has been evaluated and it was tried to prepare the next steps for a clear Spark 2.3.0 version.

How the post-groomer will run on a Kubernetes cluster can only be predicted by other Spark jobs, which should for similar because Wildfire is also based on Spark. In the beginning of chapter 3.3. has been described how to run an example Spark jar on it. There could be observed how the pods are spawning on different nodes. This confirms the advantage, which the combined usage of Spark and Kubernetes should bring with it: A fast big data analysis through Spark and a high availability and scalability of those processes through Wildfire.

To test that more detailed in the future first the Wildfire post-groomer and all its dependencies should be upgraded to Spark 2.3.0. This enables the possibility to run it without any workaround on a Kubernetes cluster.

To observe the advantages there should be more tests executed with different specifications. First the data set has to be vary with small and big data to see its behaviour. Those results should also be compared to a single node execution of the Wildfire post-groomer. Also the amount of nodes and pods should be increases step by step to figure on which settings the Wildfire post-groomer provides its best performance. Those tests can be easily executed by the automated testing software described in chapter 3.4.

Those tests can help building a larger scaled cluster for the whole Wildfire system and for optimizing the post-groomer in the future. This will help increasing the performance of Wildfire in the future and thereby providing an even faster HTAP system for clients.